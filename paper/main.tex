\documentclass{article}
\title{Spheres}
\author{Jaime E. Forero-Romero}
\begin{document}
\maketitle
\begin{abstract}

Gravity is the dominant force shaping the distribution of galaxies in
the Universe on the larges scales. Assuming the homogeneity and
isotropy of the Universe beyond a certain physical scale one can
usually approximate the local kinematic evolution of galaxy by the
matter distribution below the homogeneity scale. In this letter we
show that if the matter distribution is composed by a discrete set of
points then, due to statistical fluctuations, the influence of matter
has a measurable effect at all scales. Our results are based on
straightforward analytical considerations, monte-carlo realizations
and the analysis of cosmological N-body simulations. We discuss the
implications of these results in the interpretation of the peculiar
motion of our galaxy. We suggest possible cosmological tests of these
ideas for future observational facilities.



\end{abstract}

1. Of all the fundamental forces gravity plays the dominant role in
defining the large scale structure of the Universe. From very
homogeneous conditions gravitational instability drives the emergence
of a web-like pattern. The influence of
matter beyond that scale can be discarded on the grounds that
the net gravitational force inside an homogeneous spherical
shell is zero. [Inhomogeneity, Kirchoff's theorem] [...]



2. Large galaxy surveys help us to define different physical scales for
homogeneity[...]


3. We consider first a simple phenomenological model where matter is
homogeneously distributed over the surface of a sphere of radius
$R$. This distribution consists of a set of $N$ point masses with mass
$m$. [...] This shows that the total force, $F_{\rm T}$, inside that spherical
distribution can be expressed as follows in terms of the force $F_{\rm
m}$ produced by each point mass:

\begin{equation}
F_{\rm T } = \sqrt{N} F_{\rm m},
\end{equation}

This has been derived before [http://arxiv.org/pdf/0903.1355v1.pdf].


4. Conventional approximations consider that the net gravitational force
inside an statistically homogeneous shell matter distribution is
zero. However, by extending the result we have just derived, as long
the matter distribution is discrete, this result does not hold. We can
now extend this result for an statistically homogeneous distribution
of points.[...]

5. In the case of the actual large scale matter distribution in the
Universe, galaxies are found to have two distinct characteristics with
respect our toy model. First, the galaxies are not randomly
distributed, but clustered. Second, the galaxies span a wide range of
masses, with less massive galaxies being more common than massive
galaxies. In order to test the influence of these two carachteristics
we use a large comological N-body simulation.[...]

6. The results of the simulation [...] 

7. We also consider the variation in the direction of the net force
when the positions of the halos are perturbed within 1 $h^{-1}$ which
is the typical lnghtscale that these objects are expected to travel in
the age of the Universe.


8. The fact that our simple analytical model describes the result
obtained by simulations strenghtens our interpretation on the basis of
fundamental point processes. [...]

9. The velocity of our galaxy in the rest-frame of the cosmic
micriwave background is 627 km
s$^{-1}$. [http://arxiv.org/pdf/1109.3856v1.pdf]. It has been inferred
that the $382$ km s$^{-1}$ are induced by the mass distribution within
$R=30$ $h^{-1}$ Mpc, we call this the local component. This gives a
net result of 382 km s$^{-1}$ that must be induced by the matter
distribution from matter with positions beyond $R$, this is referred
as the tidal component.  Both components, local and tidal, are
pointing in the same direction (??? SURE ??).[...]

10. The effects of the net force imposed by the matter distribution
in the Universe could be observationally detected in systems isolated 
from massive structures.

11. Possible effects on detailed methods that seek to infer local
density modes from redshift measurements. 









\end{document}
